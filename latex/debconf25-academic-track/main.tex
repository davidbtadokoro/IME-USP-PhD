
%% bare_conf.tex
%% V1.4b
%% 2015/08/26
%% by Michael Shell
%% See:
%% http://www.michaelshell.org/
%% for current contact information.
%%
%% This is a skeleton file demonstrating the use of IEEEtran.cls
%% (requires IEEEtran.cls version 1.8b or later) with an IEEE
%% conference paper.
%%
%% Support sites:
%% http://www.michaelshell.org/tex/ieeetran/
%% http://www.ctan.org/pkg/ieeetran
%% and
%% http://www.ieee.org/

%%*************************************************************************
%% Legal Notice:
%% This code is offered as-is without any warranty either expressed or
%% implied; without even the implied warranty of MERCHANTABILITY or
%% FITNESS FOR A PARTICULAR PURPOSE! 
%% User assumes all risk.
%% In no event shall the IEEE or any contributor to this code be liable for
%% any damages or losses, including, but not limited to, incidental,
%% consequential, or any other damages, resulting from the use or misuse
%% of any information contained here.
%%
%% All comments are the opinions of their respective authors and are not
%% necessarily endorsed by the IEEE.
%%
%% This work is distributed under the LaTeX Project Public License (LPPL)
%% ( http://www.latex-project.org/ ) version 1.3, and may be freely used,
%% distributed and modified. A copy of the LPPL, version 1.3, is included
%% in the base LaTeX documentation of all distributions of LaTeX released
%% 2003/12/01 or later.
%% Retain all contribution notices and credits.
%% ** Modified files should be clearly indicated as such, including  **
%% ** renaming them and changing author support contact information. **
%%*************************************************************************


% *** Authors should verify (and, if needed, correct) their LaTeX system  ***
% *** with the testflow diagnostic prior to trusting their LaTeX platform ***
% *** with production work. The IEEE's font choices and paper sizes can   ***
% *** trigger bugs that do not appear when using other class files.       ***                          ***
% The testflow support page is at:
% http://www.michaelshell.org/tex/testflow/


\documentclass[conference]{IEEEtran}

% correct bad hyphenation here
\hyphenation{op-tical net-works semi-conduc-tor}

\begin{document}

\title{Guidelines for Boosting Long-Lasting FLOSS Contributors}


\author{
  \IEEEauthorblockN{David Tadokoro}
  \IEEEauthorblockA{University of São Paulo\\
  davidbtadokoro@usp.br}
  \and
  \IEEEauthorblockN{Paulo Meirelles}
  \IEEEauthorblockA{University of São Paulo\\
  paulormm@ime.usp.br}
}


\maketitle

\begin{abstract}
  Free Libre and Open Source Software (FLOSS) development is a validated
  approach to producing cutting-edge software solutions used by governments,
  companies, and society at large. At the core of all FLOSS software projects
  are communities composed primarily of developers who evolve and maintain the
  software while devising rules for the development process. A common problem is
  the renewal and/or scaling of the workforce, which is not about encouraging
  waves of new and sporadic contributors but about fostering long-lasting ones
  that can have a more profound impact on the project. This work presents
  guidelines for mentoring students to build skills and gain experience
  essential to becoming valuable assets in most FLOSS ecosystems. Beyond a
  reasonable software development background, previous experience in FLOSS
  development is not a requirement. These guidelines are a product of a 15-weeks
  university course where students had to (1) learn the fundamentals of Linux
  kernel development and send patches to a subsystem, (2) contribute to
  supporting tools to the GNU/Linux ecosystem, (3) introduce software packaging
  in the context of Debian, and finally (4) contribute to a chosen FLOSS
  project. The method reproduces the natural way a “self-taught” contributor
  would enter a FLOSS ecosystem (heavily inspired by how Debian does it), though
  in a more focused and immersive environment. Through in-loco workshops,
  accessible and knowledgeable mentors, and close and constant monitoring, we
  simulated a smaller FLOSS community where students (inexperienced
  contributors) learned from mentors (maintainers); we argue that this simulated
  community can be more efficient than a real one as feedback is faster and more
  adapted to each contributor, and concepts can be abstracted and simplified for
  easier absorption. Our results show that the method fostered (1) developers
  who are more confident in contributing to any FLOSS project and beyond to the
  “one-commit contributor” and (2) the enhancement of fundamental hard skills
  (like Git and Web/email-based contribution models) and soft skills (like
  communication and feedback assimilation) for any FLOSS project. Even though
  deep technical knowledge becomes mandatory for every contributor in a specific
  project, we claim that implementing the proposed guidelines can quickly
  nurture tens of developers with a solid base for becoming long-lasting
  contributors to FLOSS.
\end{abstract}


\IEEEpeerreviewmaketitle


\section{Introduction}
This demo file is intended to serve as a ``starter file''
for IEEE conference papers produced under \LaTeX\ using
IEEEtran.cls version 1.8b and later.

I wish you the best of success.

\hfill mds

\hfill August 26, 2015

\subsection{Subsection Heading Here}
Subsection text here.


\subsubsection{Subsubsection Heading Here}
Subsubsection text here.


\section{Conclusion}
The conclusion goes here~\cite{test-reference}.


\section*{Acknowledgment}

The authors would like to thank...


\bibliographystyle{IEEEtran}
\bibliography{references}

% that's all folks
\end{document}


