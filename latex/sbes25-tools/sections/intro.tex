\section{Introduction}

As a software solution, the Linux kernel is highly customizable and is on the
cutting edge of diverse fields, including Internet infrastructure, embedded
systems, cloud computing, and machine learning, making it critical for solving
real-world problems of society at large. 

When looking at the Linux kernel community from its data points, it is safe to
conclude that this project is one of the most thriving Free/Libre Open-Source
Software (FLOSS) projects by many metrics, such as the volume of volunteer
developers worldwide or the number of companies that invest in the project. The
Linux community has brought many people worldwide from different cultures,
beliefs, time zones, and others together to improve the project.\looseness=-1

Even though the project has been successful for over thirty years, it is still
evolving from the software engineering perspective. For example, due to its
size, the kernel requires developers to follow multiple processes and practices
that compose many workflows, some of which have associated bottlenecks that slow
down development; in this sense, newcomers need some time to get used to the
basic workflows, but even more experienced developers forget some aspects of the
development model and make mistakes. This is natural and, in some way, expected
when considering the vastness of the Linux kernel.

To mitigate workflow bottlenecks of large FLOSS projects, communities create
tools to streamline and automate their tasks. In the case of Linux, tools like
\texttt{get\_maintainers.pl}, \texttt{checkpatch.pl}, and
\texttt{git-send-email} help in the tasks of finding the correct recipients to
send contributions, checking for code style violations, and distributing patches
(code changes) via email, respectively. It is worth noting that configuring
these tools is often a complex and time-consuming chore by itself.

The main point is that multiple tools maintained inside and outside the Linux
project support the many kernel workflows. Beyond more consolidated and widely
adopted tools, for instance, the ones maintained inside the project's codebase
in the \texttt{scripts/} directory, it is commonplace for practitioners to
develop \textit{ad-hoc} scripts highly coupled with their development contexts
as solutions to gaps in the workflows not yet covered by existing tools or to
configuration overheads. As the primary goal of Linux developers is not to
develop and maintain supporting tools, these local solutions (although helpful)
duplicate community efforts and pulverize their resources, which could be spent
on fostering robust, high-quality collaborative tools.\looseness=-1

In other words, although many solutions cover many parts of the kernel
workflows, they are not unified, sometimes involve considerable effort to
configure, and are often unknown outside specific development contexts. All
Linux developers - contributors or maintainers, newcomers or veterans - would
benefit from a project that robustly integrates these solutions and implements
novel ones, presenting a hub-like interface that comprehensively abstracts the
kernel workflows.\looseness=-1

With all those ideas in mind, and considering that the described tools are
invaluable to maintaining a large-scale project like the Linux kernel, to the
point that they are crucial for its long-term
sustainability~\cite{corbet-gregkh2017-linuxreport}, this paper presents a set
of tools in a unified interface called \textit{Kworkflow} (kw), that streamlines
and automates the many kernel workflows while enforcing the project rules.

%%% TODO: SHOULD WE INCLUDE THE MICRO-DEVOPS DISCUSSION? %%%
% With all those ideas in mind, this paper argues for the rise of a new set of
% micro-DevOps tools that emerge from complex projects focused on ensuring that
% the project is maintainable and scalable for all levels of developers. This work
% was conducted over the years by creating an actual tool that intends to unify
% the kernel workflow in a single tool, enforcing the project rules in a
% low-effort way for experienced developers and newcomers.
